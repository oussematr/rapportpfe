\documentclass[a4paper,12pt]{report}
\usepackage{graphicx}
\graphicspath{ {figures/} }
\usepackage{array}
\usepackage{lipsum}
\usepackage{float}
\usepackage{siunitx}
\usepackage[T1]{fontenc}
\usepackage[utf8]{inputenc}
\usepackage[belowskip=-10pt,aboveskip=10 pt]{caption}
\usepackage{setspace}
\doublespacing
\setlength{\intextsep}{20pt plus 2pt minus 2pt}
\setlength{\belowcaptionskip}{-15pt}
\usepackage{natbib}
\usepackage[french]{babel}
\usepackage{lastpage}
\usepackage{fancyhdr}

\renewcommand{\headrulewidth}{1mm}
\renewcommand{\footrulewidth}{1mm}
\begin{document}
	\pagestyle{empty}
	% \tableofcontents
	% \listoffigures
	% \listoftables

	\section*{Introduction générale}
l'émergence de nouvelles applications, l'expansion rapide des réseaux et les impératifs opérationnels actuels imposent des exigences de plus en plus strictes en matière de transmission de données sur de longues distances. Cette évolution pousse les opérateurs de réseaux à repenser fondamentalement la conception des réseaux étendus.

Les réseaux étendus traditionnels sont généralement conçus pour fournir une livraison au meilleur effort, sans garantir la satisfaction des exigences spécifiques des applications. Toutefois, cette approche présente des limites lorsqu'il s'agit de répondre aux besoins de nouvelles applications émergentes, telles que les jeux en ligne et la télémédecine, qui requièrent des réseaux à faible latence pour des opérations en temps réel. Cette incompatibilité entre les exigences croissantes des applications et la méthode traditionnelle de livraison au meilleur effort rend difficile la fourniture de services satisfaisants dans ces domaines.
De plus, les fournisseurs de services Internet sont confrontés à la nécessité constante de déployer de nouveaux services sur leurs réseaux. Avec le temps, des centaines de milliers de dispositifs réseau sont déployés sur ces réseaux étendus, chacun nécessitant une configuration spécifique au fournisseur. Pour mettre à jour efficacement le réseau, les opérateurs doivent souvent effectuer ces configurations manuellement sur une période prolongée, ce qui ralentit le développement des activités. La croissance rapide du réseau complique davantage ce processus de configuration et entraîne l'apparition d'erreurs supplémentaires.

De plus, les exigences changeantes des applications agissent comme un catalyseur de cette situation. Par exemple, certains événements Internet peuvent nécessiter une augmentation temporaire de la bande passante pour faire face à une augmentation prévue du trafic de données, ce qui demande beaucoup d'efforts et peut entraîner plusieurs erreurs.

Face à ces défis, il est essentiel d'adopter de nouvelles technologies pour moderniser les réseaux étendus. Le SD-WAN, ou réseau étendu défini par logiciel, émerge comme une solution prometteuse pour répondre aux besoins du réseau étendu moderne. Grâce au SD-WAN, les entreprises peuvent bénéficier de plusieurs avantages significatifs.
\end{document}